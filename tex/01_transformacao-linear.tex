%==============================================================================
% Programação Linear
% 25/08/2021
%==============================================================================

\chapter{Programação Linear}

\section{Discussão da Solução Ótima} %-----------------------------------------

Antes de entrarmos na discussão da solução ótima faremos alguns comentários sobre
o \textbf{Método Simplex}.

O Método Simplex faz uso do algorítimo \textbf{Simplex Duas Fases}.
A Fase I é requerida quando após se acrescentar as variáveis de folga e de excesso
às inequações, isto é, passar o problema à forma padrão, não se vislumbra uma
solução básica factível de pronto.
O objetivo da Fase I é obter uma solução básica factível inicial.
A Fase II é o algorítimo propriamente dito, que para começar as iterações requer
uma solução básica factível inicial.
Os problemas cujas restrições são inequações do tipo ``menor ou igual'', e lembrando
que o vetor $ b $ é sempre positivo, têm sempre uma solução básica factível
inicial.
Nos outros casos, em geral, não tem uma solução básica factível inicial, logo,
necessitam de proceder a Fase I.

Para discussão da solução ótima, suporemos que se tem uma solução básica factível
inicial.
Logo, vamos utilizar a Fase II.

Tanto na Fase I, como na Fase II, a cada iteração verifica-se se a solução básica
atual é ótima.
Se é ótima, fim.
Se não é ótima, se vai para o critério de escolha da variável não básica que deve 
ser a candidata para entrar na base.
Escolhida a variável não básica candidata a entrar na base vem o critério da 
variável básica que sairá da base.

Pode ocorrer empate, tanto para a escolha da variável entrante, como para escolha
da que vai sair.
Ditas essas coisas em linhas gerais, vamos aos detalhes.

Voltemos ao sistema de equações,

\begin{align*}
  \min \quad        & z = c_I \inversaA b + \left[c_J - c_I \inversaA A^{J}\right] x_J\\
  \text{s.a.} \quad & x_I + \inversaA A^{J} x_J = \inversaA b \\
                    & x_I, x_J \geq 0
\end{align*}

Na Função Objetivo a parte variável é a segunda parcela do segundo membro,

\[
  \left[ c_J - c_I \inversaA A^{J} \right]  x_J
\]

Suponha que a solução básica atual é ótima.
Isto significa que o valor de $ z $, que é $ c_I \inversaA b $, não tem mais 
como diminuir, ou seja, se algum $ x_j $ (variável não básica) aumentar de valor,
o valor de $ z $ aumentará.
Isto implica que o coeficiente $ \widehat{c}_j $ de $ x_j $ é positivo.
Por outro lado, se todos os $ \widehat{c}_j > 0 $, qualquer variável $ x_j $
que aumentar de valor fará o valor de $ z $ também aumentar, logo, a solução 
atual é ótima.

Conclusão:

\begin{center}
\fbox
{
  \begin{minipage}{0.7\linewidth}
    A solução básica atual é ótima e única se, e somente se, $ \widehat{c}_j > 0 $,
    $ \forall j \in J $.
  \end{minipage}
}
\end{center}

Se a solução básica atual é ótima e algum $ \widehat{c}_j = 0 $, haverá infinitas 
soluções ótimas, pois para cada valor positivo que $ x_j $ assumir corresponderá 
a novos valores das variáveis básicas, mas o valor de $ z $ permanece o mesmo, 
ou seja, $ c_I \inversaA b $, pois $ \widehat{c}_j x_j = 0 $.
Para saber os valores das variáveis básicas para cada valor positivo assumido
por $ x_j $, observe a equação de bloqueio:

\[
  x_I + \widehat{A}^j x_j = \widehat{b}
\]

Agora cabe uma pergunta: \textit{Neste caso há outra solução básica ótima?}
Mais uma vez a resposta é dada pela equação de bloqueio.

Se algum $ \widehat{a}_ij > 0 $, haverá outra solução básica ótima, pois a 
variável básica da linha $ i $ irá para zero quando $ x_j $ aumentar de valor.
Caso haja mais de um $ \widehat{a}_ij > 0 $, as variáveis básicas das respectivas
linhas irão para zero.
Mas, se uma variável atingiu o valor zero, o $ x_j $ não pode mais aumentar de 
valor, caso contrário, a variável básica torna-se-á negativa.
Logo, $ x_j $ será bloqueado quando a primeira variável básica atingir o valor
zero.
O valor de $ x_j $ que faz a variável básica $ x_{B_i} = 0 $ é dado por:

\[
  \frac{\widehat{b}_i}{\widehat{a}_{ij}}  
\]

Assim, se houver mais de uma variável básica indo para zero, deve se escolher
aquela que primeiro atingir zero.
Para isso, calcula-se:

\[
  \min_{i}  
  \left\{\, 
    \frac
    {
      \widehat{b}_i
    }
    {
      \widehat{a}_{ij}
    }; 
    \quad 
    \widehat{a}_{ij} > 0 
  \,\right\}
\]

\begin{obs}
  Se duas ou mais variáveis básicas atingirem zero simultaneamente, estamos 
  diante de uma solução básica degenerada.
  Mas, só uma delas sai da base e $ x_j $ entra no lugar da que saiu.
\end{obs}

O cáculo iterativo para se chegar a nova solução básica é transformar a coluna
$ \widehat{A}^j $ na coluna da matriz identidade que está com a variável básica
que está saindo da base.
Se $ x_{B_i} $ é a variável básica que está saindo da base, o primeiro cálculo
a ser feito é dividir toda linha $ i $ por $ \widehat{a}_{ij} $.
A partir daí, por meio de operações elementares sobre as linhas, é transformar
em zero o $ \widehat{c}_j $ e todos os $ \widehat{a}_{kj} $, com $ k \in I $ e 
$ k \neq i $.
Feito isto, obteve-se a nova solução básica.
Agora, volta-se ao teste de otimalidade.

\section{Montagem do Quadro Simplex} %-----------------------------------------
