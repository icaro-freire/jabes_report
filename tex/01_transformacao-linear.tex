%==============================================================================
% Programação Linear
% 25/08/2021
%==============================================================================

\chapter{Programação Linear}

\section{Discussão da Solução Ótima} %-----------------------------------------

Antes de entrarmos na discussão da solução ótima faremos alguns comentários sobre
o \textbf{Método Simplex}.

O Método Simplex faz uso do algorítimo \textbf{Simplex Duas Fases}.
A Fase I é requerida quando após se acrescentar as variáveis de folga e de excesso
às inequações, isto é, passar o problema à forma padrão, não se vislumbra uma
solução básica factível de pronto.
O objetivo da Fase I é obter uma solução básica factível inicial.
A Fase II é o algorítimo propriamente dito, que para começar as iterações requer
uma solução básica factível inicial.
Os problemas cujas restrições são inequações do tipo ``menor ou igual'', e lembrando
que o vetor $ b $ é sempre positivo, têm sempre uma solução básica factível
inicial.
Nos outros casos, em geral, não tem uma solução básica factível inicial, logo,
necessitam de proceder a Fase I.

Para discussão da solução ótima, suporemos que se tem uma solução básica factível
inicial.
Logo, vamos utilizar a Fase II.

Tanto na Fase I, como na Fase II, a cada iteração verifica-se se a solução básica
atual é ótima.
Se é ótima, fim.
Se não é ótima, se vai para o critério de escolha da variável não básica que deve 
ser a candidata para entrar na base.
Escolhida a variável não básica candidata a entrar na base vem o critério da 
variável básica que sairá da base.

Pode ocorrer empate, tanto para a escolha da variável entrante, como para escolha
da que vai sair.
Ditas essas coisas em linhas gerais, vamos aos detalhes.

Voltemos ao sistema de equações,

\begin{align*}
  \min \quad        & z = c_I \inversaA b + \left[c_J - c_I \inversaA A^{J}\right] x_J\\
  \text{s.a.} \quad & x_I + \inversaA A^{J} x_J = \inversaA b \\
                    & x_I, x_J \geq 0
\end{align*}

Na Função Objetivo a parte variável é a segunda parcela do segundo membro,

\[
  \left[ c_J - c_I \inversaA A^{J} \right]  x_J
\]

Suponha que a solução básica atual é ótima.
Isto significa que o valor de $ z $, que é $ c_I \inversaA b $, não tem mais 
como diminuir, ou seja, se algum $ x_j $ (variável não básica) aumentar de valor,
o valor de $ z $ aumentará.
Isto implica que o coeficiente $ \widehat{c}_j $ de $ x_j $ é positivo.
Por outro lado, se todos os $ \widehat{c}_j > 0 $, qualquer variável $ x_j $
que aumentar de valor fará o valor de $ z $ também aumentar, logo, a solução 
atual é ótima.

Conclusão:

\begin{center}
\fbox
{
  \begin{minipage}{0.7\linewidth}
    A solução básica atual é ótima e única se, e somente se, $ \widehat{c}_j > 0 $,
    $ \forall j \in J $.
  \end{minipage}
}
\end{center}

Se a solução básica atual é ótima e algum $ \widehat{c}_j = 0 $, haverá infinitas 
soluções ótimas, pois para cada valor positivo que $ x_j $ assumir corresponderá 
a novos valores das variáveis básicas, mas o valor de $ z $ permanece o mesmo, 
ou seja, $ c_I \inversaA b $, pois $ \widehat{c}_j x_j = 0 $.
Para saber os valores das variáveis básicas para cada valor positivo assumido
por $ x_j $, observe a equação de bloqueio:

\[
  x_I + \widehat{A}^j x_j = \widehat{b}
\]

Agora cabe uma pergunta: \textit{Neste caso há outra solução básica ótima?}
Mais uma vez a resposta é dada pela equação de bloqueio.

Se algum $ \widehat{a}_ij > 0 $, haverá outra solução básica ótima, pois a 
variável básica da linha $ i $ irá para zero quando $ x_j $ aumentar de valor.
Caso haja mais de um $ \widehat{a}_ij > 0 $, as variáveis básicas das respectivas
linhas irão para zero.
Mas, se uma variável atingiu o valor zero, o $ x_j $ não pode mais aumentar de 
valor, caso contrário, a variável básica torna-se-á negativa.
Logo, $ x_j $ será bloqueado quando a primeira variável básica atingir o valor
zero.
O valor de $ x_j $ que faz a variável básica $ x_{B_i} = 0 $ é dado por:

\[
  \frac{\widehat{b}_i}{\widehat{a}_{ij}}  
\]

Assim, se houver mais de uma variável básica indo para zero, deve se escolher
aquela que primeiro atingir zero.
Para isso, calcula-se:

\[
  \min_{i}  
  \left\{\, 
    \frac
    {
      \widehat{b}_i
    }
    {
      \widehat{a}_{ij}
    }; 
    \quad 
    \widehat{a}_{ij} > 0 
  \,\right\}
\]

\begin{obs}
  Se duas ou mais variáveis básicas atingirem zero simultaneamente, estamos 
  diante de uma solução básica degenerada.
  Mas, só uma delas sai da base e $ x_j $ entra no lugar da que saiu.
\end{obs}

O cáculo iterativo para se chegar a nova solução básica é transformar a coluna
$ \widehat{A}^j $ na coluna da matriz identidade que está com a variável básica
que está saindo da base.
Se $ x_{B_i} $ é a variável básica que está saindo da base, o primeiro cálculo
a ser feito é dividir toda linha $ i $ por $ \widehat{a}_{ij} $.
A partir daí, por meio de operações elementares sobre as linhas, é transformar
em zero o $ \widehat{c}_j $ e todos os $ \widehat{a}_{kj} $, com $ k \in I $ e 
$ k \neq i $.
Feito isto, obteve-se a nova solução básica.
Agora, volta-se ao teste de otimalidade.

\section{Montagem do Quadro Simplex} %-----------------------------------------

Toda a discussão sobre a solução ótima e pivoteamento foram realizadas com o P.L.
na forma:

\begin{align*}
  \min        \quad & z = c_I \inversaA b + \left[c_J - c_I \inversaA A^J \right] x_J \\
  \text{s.a.} \quad & x_I + \inversaA A^J x_J = \inversaA b \\
                    & x_I, x_J \geq 0
\end{align*}

Para montagem do quadro, toda expressão que tiver variável colocamos no primeiro
membro das equações, e as constantes, no segundo.
Mas, para manter a discussão sem alterações, sobre o sinal de $ \widehat{c}_J $,
colocamos a F.O. na seguinte forma:

\[
  \left[c_J - c_I \inversaA A^J \right] x_J - z = -c_I \inversaA b
\]

E o quadro do Simplex é o seguinte:

\begin{table}[!htbp]
  \centering
  \begin{tabular}{c|ccc|c}
    Min         & $x_I$        & $x_J$                       &  $z$  & \\ \hline
    Var. Bás.   & $0$          & $c_J - c_I \inversaA A^{J}$ & $-1$  & $-c_I \inversaA b$\\ \hline
    $x_I$       & $\mathbb{I}$ & $\inversaA A^J$             & $0$   & $\inversaA b$ \\ \hline
                &              & $x_I, x_J \geq 0$           &       &
  \end{tabular}
\end{table}

Ou fazendo uso da notação simplificada:

\begin{table}[!htbp]
  \centering
  \begin{tabular}{c|ccc|c}
    Min       & $x_I$        & $x_J$           & $z$  & \\ \hline
    Var. Bás. & $0$          & $\widehat{c}_J$ & $-1$ & $-c_I \inversaA b$ \\ \hline
    $x_I$     & $\mathbb{I}$ & $\widehat{A}^J$ & $0$  & $\widehat{b}$ \\ \hline
              &              & $x_I, x_J$      &      &
  \end{tabular}
\end{table}

Na prática, na primeira linha, as variáveis são dispostas na ordem crescente dos
índices; e, na última linha, não colocamos.

Os procedimentos e discussões sobre a solução ótima são os mesmso tanto para a 
Fase I, como para a Fase II.

Vejamos um exemplo que não requer a Fase I.

\begin{exemplo}
  Resolva o P.L. abaixo:
  \begin{align*}
    \min        \quad & z = -3 x_1 + 5 x_2 - 2 x_3 \\
    \text{s.a.} \quad & 
                        \begin{aligned}[t]
                          2 x_1 + 2 x_2 + 4 x_3 &\leq 12 \\
                          - x_1 + 3 x_2 + 2 x_3 &\leq 6  \\
                          2 x_1 + 4 x_2 -   x_3 &\leq 4
                        \end{aligned}\\
                      & x_1, x_2, x_3 \geq 0.
  \end{align*}
\end{exemplo}

Primeiro passa o problema para forma padrão, acrescentando-se as variáveis de 
folga.

\begin{align*}
  \min        \quad & z = -3 x_1 + 5 x_2 - 2 x_3 \\
  \text{s.a.} \quad & 
                      \begin{aligned}[t]
                        2 x_1 + 2 x_2 + 4 x_3 + x_4 \phantom{+ x_5 + x_6} &= 12 \\
                        - x_1 + 3 x_2 + 2 x_3 \phantom{+ x_4} + x_5 \phantom{+ x_6}&= 6  \\
                        2 x_1 + 4 x_2 -   x_3 \phantom{+ x_4 + x_5} + x_6 &= 4
                      \end{aligned}\\
                    & x_1, x_2, x_3, x_4, x_5 \geq 0.
\end{align*}

Note que há uma solução básica factível inicial: 
\begin{itemize}
  \item \textbf{Variáveis Básicas:} $ x = 12, x_5 = 6 \text{ e } x = 4 $;
  \item \textbf{Variáveis não Básicas:} $ x_1 = x_2 = x_3 = 0$;
  \item \textbf{Valor de F.O.}: $ z = 0 $.
\end{itemize}

Montando o quadro Simplex:

\begin{table}[!htbp]
  \centering
  \begin{tabular}{c|ccccccc|c}
    Min             & $\setaEntra{1}$   & $x_2$ & $x_3$ & $x_4$ & $x_5$ & $x_6$ & $z$  &      \\ \hline
    Var. Bás.       & $-3$              & $5$   & $-2$  & $0$   & $0$   & $0$   & $-1$ & $0$  \\ \hline           
    $x_4$           & $ 2$              & $2$   & $4$   & $1$   & $0$   & $0$   & $0$  & $12$ \\
    $x_5$           & $-1$              & $3$   & $2$   & $0$   & $1$   & $0$   & $0$  & $6$  \\
    $\setaSai{6}$   & $\circulo{2}$     & $4$   & $-1$  & $0$   & $0$   & $1$   & $0$  & $4$
  \end{tabular}
\end{table}

Neste caso, em que as inequações são todos do tipo ``$\leq$'', o quadro sempre está
preparado, istó é, pronto para discussão se a solução  básica atual é ótima ou
não.

Há dois $\widehat{c} < 0$, o que indica que a solução básica atual não é ótima 
($\widehat{c}_7 = -3$ e $\widehat{c}_3 = -2$).
O mais negativo corresponde à variável não básica que deve ser a candidata a 
entrar na base.
Logo, $ x_1 $ é a candidata a entrar  na base.
Agora, examinaremos o $ \widehat{A}^{1} $.

\[
  \widehat{A}^{1} = 
    \begin{bmatrix}
      2 \\
      -1 \\
      2
    \end{bmatrix}
\]

Como há elementos positivos, há outra solução básica factível que melhora a F.O.
($\widehat{a}_{11} = 2$ e $\widehat{a}_{31} = 2$).

A equação de bloqueio é 
\[
  \begin{bmatrix}
    x_4 \\
    x_5 \\
    x_6
  \end{bmatrix}  
  +
  \begin{bmatrix}
    2 \\
    -1\\
    2
  \end{bmatrix}
  = 
  \begin{bmatrix}
    12 \\
    6 \\
    4
  \end{bmatrix},
\]
 e o cálculo prático para se saber que variável  básica sai da base é 
 \[
   \Min_{i} \left\{ \frac{\widehat{b}_i}{\widehat{a}_{ij}}; \;\; \widehat{a}_{ij} > 0 \right\}
 \]

 Assim, tem-se o seguinte:
 \[
   \Min\left\{\frac{12}{2}, \frac{4}{2}\right\}  = \frac{4}{2} = 2.
 \]

 Significa que $ x_6 $ deve sair da base, e $ x_1 $ entra na base com valor 2.
 
 Observe: se $ x_1 = 2 $, então $ x_6 = 0, x_4 = 8 \text{ e } x_5 = 8 $.

 A coluna $ \widehat{A}^{1} $ deve ser transformada na coluna 
 \[
   \begin{bmatrix} 
    0 \\ 
    0 \\ 
    1 
   \end{bmatrix} 
\] 
que é a atual coluna de $ x_6 $.

O novo quadro é:

\begin{table}[!htbp]
  \centering
  \begin{tabular}{c|ccccccc|c}
    Min           & $x_1$ & $x_2$ & $\setaEntra{3}$ & $x_4$ & $x_5$ & $x_6$ & $z$  &     \\ \hline
    Var. Bás.     & $0$   & $11$  & $-7/2$          & $0$   & $0$   & $3/2$ & $-1$ & $6$ \\ \hline
    $\setaSai{4}$ & $0$   & $-2$  & $\circulo{5}$   & $1$   & $0$   & $-1$  & $0$  & $8$ \\
    $x_5$         & $0$   & $5$   & $3/2$           & $0$   & $1$   & $1/2$ & $0$  & $8$ \\
    $x_1$         & $1$   & $2$   & $-1/2$          & $0$   & $0$   & $1/2$ & $0$  & $2$ 
  \end{tabular}
\end{table}

A solução básica atual não é ótima: $ \widehat{c}_3 = -7/2 $.
Único negativo, logo, $ x_3 $ é camdidato a entrar na base.

O elemento do encontro da coluna da variável que entra com a linha da variável 
que sai é chamado \textbf{pivô}.
Ele é que será transformado em ``1'' e todos os cálculos são realizados por meio 
dele.

Da \textbf{Equação de Bloqueio} 
\[
  \begin{bmatrix}
    x_4 \\
    x_5 \\
    x_1
  \end{bmatrix}  
  + \begin{bmatrix}
    5    \\
    3/2  \\
    -1/2 \\
  \end{bmatrix}
  = 
  \begin{bmatrix}
    8 \\
    8 \\
    2
  \end{bmatrix}
\]
se retira o cálculo prático, do qual se obtém duas informações:
\begin{enumerate}
  \item[(i)] com que valor $ x_3 $ \textbf{entra};
  \item[(ii)] e que variável \textbf{sai} da base.
\end{enumerate}

Como
\[
  Min\left\{\frac{8}{5}, \frac{8}{3/2}\right\}  = \frac{8}{5},
\]
logo, $x_4$ sai da base.

\begin{table*}[!htbp]
  \centering
  \begin{tabular}{c|ccccccc|c}
    Min       & $x_1$ & $x_2$  & $x_3$ & $x_4$   & $x_5$ & $x_6$  & $z$  &        \\ \hline
    Var. Bás. & $0$   & $48/5$ & $0$   & $7/10$  & $0$   & $4/5$  & $-1$ & $58/5$ \\ \hline 
    $x_3$     & $0$   & $-2/5$ & $1$   & $1/5$   & $0$   & $-1/5$ & $0$  & $8/5$  \\
    $x_5$     & $0$   & $28/5$ & $0$   & $-3/10$ & $1$   & $4/5$  & $0$  & $28/5$ \\
    $x_1$     & $1$   & $9/5$  & $0$   & $1/10$  & $0$   & $2/5$  & $0$  & $14/5$ 
  \end{tabular}
\end{table*}

Todos os $ \widehat{c}_J > 0 $, logo a solução básica atual é ótima e única.

\begin{itemize}
  \item $ \widehat{c}_2 = 48/5, \widehat{c}_4 = 7/10 \;\text{ e }\; \widehat{c}_6 = 4/5 $
  \item o valor ótimo de $ z $ é $ -58/5 $
  \item variáveis básicas: $ x_3 = 8/5, x_5 = 28/5 \;\text{ e }\; x_1 = 14/5 $
  \item variáveis não básicas: $ x_2 = x_4 = x_6 = 0 $
\end{itemize}

\section{Fase I} %-------------------------------------------------------------

