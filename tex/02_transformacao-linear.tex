%==============================================================================
% Programação Linear II
% 04/08/2021
%==============================================================================

\chapter{Programação Linear II}

\section{Solução Básica} %=====================================================

Seja $ A $ uma matriz $ m \times n $ de \textit{rank} $ m $ e considere o 
sistema $ Ax = b $ de equações lineares.
Como $ A $ tem \textit{rank} $ m $ é possível extrair de $ A $ $ m $ vetores
coluna linearmente independentes.
Com isto, é possível tirar os valores das $ m $ variáveis associadas a esses
$ m $ vetores colunas em função das $ n - m $ variáveis restantes.

Uma solução básica é obtida fazendo essas $ n - m $ variáveis restantes 
assumirem o valor zero.

\begin{exemplo}
  Considere o sistema 
  \[
   \systeme{3x_1-2x_2+5x_3+6x_4 = 4, 2x_1+3x_2+4x_3-2x_4=6}
  \]
\end{exemplo}

A matriz aqui é 
$
  \begin{bmatrix} 
   3 & -2 & 5 & 6 \\ 
   2 &  3 & 4 & -2
  \end{bmatrix}
$.
Claro que $ A $ tem \textit{rank} 2.
Em particular, quaisquer dois vetores coluna escolhidos em $ A $ são linearmente
independentes.
Escolhemos as colunas 2 e 3, o que significa que vamos tirar os valores das 
variáveis $ x_2 $ e $ x_3 $ em termos das variáveis $ x_1 $ e $ x_2 $.
Há diversas maneiras de se resolver, mas vamos fazer uso de inversão de matrizes,
no caso, calculando a inversa da matriz $\begin{bmatrix} -2 & 5 \\ 3 & 4\end{bmatrix}$.

A inversa\footnote{Para matrizes $ 2 \times 2 $ há uma maneira prática de se 
calcular a inversa. 
Se $ A = \begin{bmatrix} a & b \\ c & d\end{bmatrix} $, então a sua inversa é
$ \dfrac{1}{ad - bc}\begin{bmatrix} d & -b \\ -c & a\end{bmatrix} $
} é 
$
  \dfrac{-1}{23}
  \begin{bmatrix} 
    4 & -5 \\ 
   -3 & -2
  \end{bmatrix}
$.
Agora é só multiplicar todo o sistema por essa inversa.

Podemos fazer o cálculo só com as matrizes para economizar tempo, mas 
acrescentando a coluna independente, $ \begin{bmatrix} 4 \\ 6\end{bmatrix}$.

{\scriptsize
\begin{align*}
  -\frac{1}{23} 
  \begin{bmatrix} 
    4 & -5 \\ 
    &\\
   -3 & -2
  \end{bmatrix} 
  \begin{bmatrix}[cccc|c]
   3 &-2 & 5 &  6 & 4 \\
    &&&&\\
   2 & 3 & 4 & -2 & 6
  \end{bmatrix}
  &=
  -\frac{1}{23}
  \begin{bmatrix}[cccc|c]
     2 & -23 &   0 &  34 & -14 \\
     &&&&\\
   -13 &   0 & -23 & -14 & -24
  \end{bmatrix}\\ \\
  &= 
  \begin{bmatrix}[cccc|c]
   -\frac{2}{23}  & 1 & 0 & -\frac{34}{23} & \frac{14}{23} \\
                  &   &   &                &               \\
    \frac{13}{23} & 0 & 1 &  \frac{14}{23} &\frac{24}{23}
  \end{bmatrix}
\end{align*}
}

Portanto, \systeme*{x_2 = \frac{14}{23} + \frac{2}{23}x_1+\frac{34}{23}x_4, x_3 = \frac{24}{23} + \frac{13}{23}x_1+\frac{14}{23}x_4}.

Tomando $ x_1 = x_4 = 0 $, obtém-se a solução básica referente a $ x_2 $ e 
$ x_3 $:

\[
\systeme*{x_2 = \frac{14}{23}, x_3 = \frac{24}{23}}.
\]

Em Programação Linear há um adicional a mais com relação às variáveis: elas 
devem ser maiores ou iguais a zero.
É uma restrição a mais que não interfere na definição de solução básica.
Então, na linguagem  de P.L., o que se diz é que, se os valores básicos 
encontrados são maiores ou iguais a zero, diz-se que é uma 
\textbf{solução básica factível}.
Caso contrário, diz-se \textbf{solução básica infactível}.
O sistema proposto é:

\[
  Ax = b, \qquad x \geq 0.
\]

Outro detalhe importante é que uma solução básica pode ser identificada pelos
índices das variáveis que não assumirão o valor zero.
Essas são denominadas de \textbf{variáveis básicas}, e aquelas que assumirão o
valor zero são denominadas de \textbf{variáveis não básicas}.
Também são ditas que essas $ m $ variáveis básicas estão na base e as não básicas
estão fora da base.
Portanto, os índices das variáveis básicas, dizem quem é a solução básica.
Mas, para completar essa identificação é necessário saber em que linha ou 
equação está a variável básica.
Isto impõe que os índices devem estar ordenados na mesma sequência em que as
correspondentes variáveis aparecem nas linhas.
Assim, o primeiro índice deve ser da variável básica que está na primeira linha,
o segundo índice, da variável básica que está na segunda linha e assim por 
diante.
Dessa forma, defini-se um conjunto ordenados dos índices das variáveis básicas,
que será denotado por $ I $; e, um outro conjunto dos índices das variáveis não
básicas, denotado por $ J $.

No exemplo dado: $ I = \{\,2, 3\,\} $ e $ J = \{\,1. 4\,\} $.

De agora em diante, uma solução básica será identificada pelo conjunto $ I $.

\begin{exemplo}
  Dado o sistema 
  \[
   \systeme
   {
     x_1 + 5x_2 - x_3 =  5,
    3x_1 + 4x_2 + x_4 = 12,
    2x_1 +  x_2 - x_5 =  4
   }
  \]
  com $ x_1, x_2, x_3, x_4, x_5 \geq 0 $ e I = \{\,2, 4, 1\,\}.
\end{exemplo}


Significa que foram tirados  os valores das variáveis $ x_1 $, $ x_2 $ e $ x_3 $,
mas que a ordem em que aparecem na solução final é que $ x_2 $ está na primeira
linha, $ x_4 $ está na segunda e $ x_1 $ está na terceira linha.
Veja abaixo:

\[
  \systeme
  {
   x_2 - \frac{2}{9}x_3 + \frac{1}{9}x_5  = \frac{2}{3},
   \frac{5}{9}x_3 + x_4 + \frac{11}{9}x_5 = \frac{13}{3},
   x_1 + \frac{1}{9}x_3 - \frac{5}{9}x_5  = \frac{5}{3}
  }
\]

A solução básica é: $ x_2 = \frac{2}{3} $, $ x_4 = \frac{13}{3} $ e 
$ x_4 = \frac{5}{3} $.
As variáveis não básicas: $ x_3 = x_5 = 0 $.

A matriz dos vetores coluna extraídos da matriz \\

\[ 
  A = 
  \begin{bmatrix} 
   1 & 5 & -1 & 0 &  0 \\
   3 & 4 &  0 & 1 &  0 \\
   2 & 1 &  0 & 0 & -1
  \end{bmatrix} 
  \qquad
  \text{ é }
  \qquad
  \begin{bmatrix}
   5 & 0 & 1 \\
   4 & 1 & 3 \\
   1 & 0 & 2
  \end{bmatrix}.
\]

Observe a ordem das colunas.
A primeira coluna é a do $ x_2 $; a segunda coluna é a do $ x_4 $ e a terceira
coluna é a do $ x_1 $.
Assim, quando multiplicar a matriz estendida do sistema pela inversa de 
$
  \begin{bmatrix}
   5 & 0 & 1 \\
   4 & 1 & 3 \\
   1 & 0 & 2
  \end{bmatrix}
$,
as colunas da matriz identidade aparecerá na mesma ordem, isto é, a primeira
coluna da matriz identidade estará na coluna do $ x_2 $, a segunda, na coluna do
$ x_4 $ e a terceira, na coluna do $ x_1 $.

Mostraremos que acontece exatamente isso.

\begin{align*}
  \begin{bmatrix}[ccc|ccc]
   5 & 0 & 1 & 1 & 0 & 0 \\
   4 & 1 & 3 & 0 & 1 & 0 \\
   1 & 0 & 2 & 0 & 0 & 1
  \end{bmatrix}
  &\approx
  \begin{bmatrix}[ccc|ccc]
   1 & -1 & -2 &  1 & -1 & 0 \\
   0 &  5 & 11 & -4 &  5 & 0 \\
   0 &  1 &  4 & -1 &  1 & 1 
  \end{bmatrix} \\
  &\approx
  \begin{bmatrix}[ccc|ccc]
   1 & 0 &  2 &  0 & 0 &   1 \\
   0 & 9 &  0 & -5 & 9 & -11 \\
   0 & 0 & -9 &  1 & 0 &  -5
  \end{bmatrix} \\
  &\approx 
  \begin{bmatrix}[ccc|ccc]
   9 & 0 & 0 &  2 & 0 &  -1 \\
   0 & 9 & 0 & -5 & 9 & -11 \\
   0 & 0 & 9 &  1 & 0 &  -5
  \end{bmatrix} \\
  &\approx 
  \begin{bmatrix}[ccc|ccc]
   1 & 0 & 0 &  2/9 & 0 &  -1/9 \\
   0 & 1 & 0 & -5/9 & 1 & -11/9 \\
   0 & 0 & 1 &  1/9 & 0 &  -5/9
  \end{bmatrix}
\end{align*}

Portanto a matriz inversa é 
$
  \begin{bmatrix}
    2/9 & 0 &  -1/9 \\
   -5/9 & 1 & -11/9 \\
    1/9 & 0 &  -5/9
  \end{bmatrix}
$.

Agora, vamos multiplicar essa matriz inversa pela matriz estendida dos 
coeficientes das variáveis, isto é, $ \begin{bmatrix} A & b \end{bmatrix} $.

{\scriptsize
\[
  \begin{bmatrix}
    2/9 & 0 &  -1/9 \\
   -5/9 & 1 & -11/9 \\
    1/9 & 0 &  -5/9
  \end{bmatrix}
  \begin{bmatrix}[ccccc|c]
   1 & 5 & -1 & 0 &  0 &  5 \\
   3 & 4 &  0 & 1 &  0 & 12 \\
   2 & 1 &  0 & 0 & -1 &  4 
  \end{bmatrix}
  =
  \begin{bmatrix}[ccccc|c]
   0 & 1 & -2/9 & 0 &  1/9 &  2/3 \\
   0 & 0 &  5/9 & 1 & 11/9 & 11/3 \\
   1 & 0 &  1/9 & 0 & -5/9 &  5/3
  \end{bmatrix}
\]
}

Note que a primeira coluna da matriz identidade é a segunda coluna da matriz
acima; que a coluna de $ x_2 $.
Analogamente para as outras colunas.

A leitura das linhas, por exemplo, da segunda linha, é:
\[
  \frac{5}{9}x_3 + x_4 + \frac{11}{9}x_5 = \frac{11}{3}.
\]

Leitura dos valores básicos, lê-se direto só as variáveis básicas e seus 
respectivos valores.
\[
  x_2 = \frac{2}{3}, \quad x_4 = \frac{11}{3} \quad \text{e} \quad x_1 = \frac{5}{3}.
\]





\section{Discussão da Solução Ótima} %=========================================

Considere o P.L. $ \text{Min. } z = cx $ sujeito à $ Ax = b $, $ x \geq 0 $.

As hipóteses são as de sempre: $ A $ é uma matriz $ m \times n $ de \textit{rank}
$ m $.

O ponto de partida é que se tenha uma solução básica factível inicial.

O que vamos apresentar é o \textbf{Algorítimo Simplex Duas Fases}.
A \textit{Fase I} é aplicada quando não se tem uma solução básica factível 
inicial.
A \textit{Fase II}, que é o algorítimo propriamente dito, precisa de usa solução
básica factível inicial para começar as iterações.
Em cada solução básica factível deve se proceder a discussão sobre se essa 
solução é ótima ou não.
Se não é ótima, vem a discussão se há outra solução básica que melhore o valor
da F.O.
Se existe, fazem-se os cálculos, procedimento para se passar à nova solução 
básica.
Se não há outra solução básica, diz-se que o problema tem solução ótima 
ilimitada.

\subsection{Fase I - Procedimentos} %------------------------------------------

\begin{enumerate}
	\item Abandone temporariamente a F.O. original do problema.
 \item Na parte das restrições, acrescente uma variável artificial em cada 
  equação em que não tenha uma variável básica factível.
  Essas variáveis devem ser maiores ou iguais a zero.
 \item Crie uma F.O. artificial a ser minimizada, que é constituída pela soma 
  das variáveis artificiais que foram introduzidas nas restrições.
 \item Monte o quado do simplex.
 \item Prepare o quadro, isto é, transformar os coeficientes das variáveis 
  básicas, que estão na F.O., em zero.
 \item Comece a discussão da solução ótima.
\end{enumerate}

\begin{exemplo}
  Considere o P.L.
  \[
   \begin{array}{rrcl}
    \textrm{Max}  & z = 4x_1 + 3x_2 &      &    \\
    \textrm{s.a.} & x_1 + 5x_2      & \geq & 5  \\
                  & 3x_1 + 4x_2     & \leq & 12 \\ 
                  &     2x_1 + x_2  & \geq & 4  \\ 
                  &     x_1, x_2    & \geq & 0  
   \end{array}
  \]
\end{exemplo}

A primeira coisa a ser feita é passar o problema para forma padrão.
Isto é feito acrescentando-se as variáveis de folga e de excesso, transformando
as desigualdades em equações.

\[
  \begin{array}{rlcl}
   \textrm{Max}  & z = 4x_1 + 3x_2 &                            &    \\
   \textrm{s.a.} & \phantom{3}x_1 +  5x_2 - x_3  \phantom{-0x_4}  \phantom{-0x_5}  & =    & 5  \\
                 &           3x_1 +  4x_2            \phantom{- 0x_3} + x_4 \phantom{- 0x_5}  & =    & 12 \\ 
                 &           2x_1 +  \phantom{0}x_2  \phantom{- 0x_3}  \phantom{+0x_4} - x_5   & =    & 4  \\ 
                 &  x_1, x_2, x_3, x_4, x_5 \geq 0                    &  &   
  \end{array}
\]

Observe que há uma solução básica inicial, mas é infactível.

$ x_3 = -5 $, $ x_4 = 12 $ e $ x_5 = -4 $ são as variáveis básicas.
$ x_1 $ e $ x_2 $ são as variáveis \textbf{não} básicas e que assumem o valor
zero.

Portanto há necessidade de proceder a \textit{Fase I}.

Há necessidade de variáveis artificiais na primeira e na terceira equações.
Tem-se:
\[
  \begin{array}{lc}
   \text{Min}  & w = x_6 + x_7  \\
   \text{s.a.} & \\
               & \sysdelim..\systeme
               {
               x_1 + 5x_2 - x_3 + x_6 = 5,
               3x_1 + 4x_2 + x_4 = 12,
               2x_1 + x_2 - x_5 + x_7 = 4
               }
  \end{array}
\]

Agora se tem uma solução básica factível, mas é artificial, ela não pertence ao
problema original.

Variáveis básicas: $ x_6 = 5 $, $ x_4 = 12 $ e $ x_7 = 4 $.

Variáveis não básicas: $ x_1 = x_2 = x_3 = x_5 = 0 $.

Valor da F.O.: $ w = 9 $.

Passando ao quadro:

\begin{table}[!h]
\centering
\caption{O quadro não está preparado}
\begin{tabular}{|c|cccccccc|c|}
  \hline
  Min         & $x_1$ & $x_2$ & $x_3$ & $x_4$ & $x_5$ & $x_6$ & $x_7$ & $x_8$ &    \\
  \hline
  Var. Bas.   &   0   &   0 &   0 &   0 &   0 &   1 &   1 &  -1 &  0 \\
  \hline
  $x_6$       &   1   &   5 &  -1 &   0 &   0 &   1 &   0 &   0 &  5 \\
  $x_4$       &   3   &   4 &   0 &   1 &   0 &   0 &   0 &   0 & 12 \\
  $x_7$       &   2   &   1 &   0 &   0 &  -1 &   0 &   1 &   0 &  4 \\
  \hline
\end{tabular}
\end{table}

\begin{table}[!h]
\centering
\caption{O quadro está preparado}
\begin{tabular}{|c|cccccccc|c|}
  \hline
  Min                    & $x_1$ & \textcolor{red}{$x_2$} & $x_3$ & $x_4$ & $x_5$ & $x_6$ & $x_7$ & $x_8$ &     \\
  \hline
  Var. Bas.              &   -3  &   -6                   &   1   &   0   &   1   &   0   &   0   &  -1   &  -9 \\
  \hline
  \textcolor{red}{$x_6$} &   1   &   \textcolor{red}{5}   &  -1   &   0   &   0   &   1   &   0   &   0   &   5 \\
  $x_4$                  &   3   &   4                    &   0   &   1   &   0   &   0   &   0   &   0   &  12 \\
  $x_7$                  &   2   &   1                    &   0   &   0   &  -1   &   0   &   1   &   0   &   4 \\
  \hline
\end{tabular}
\end{table}

A equação de bloqueio é:\;  $ x_{I} + \widehat{A}^{j}x_{J} = \widehat{b} $.

O cálculo prático é:\;  $ \underset{i}{\text{Min}} \left\{\,\frac{\widehat{b}_i}{\widehat{a}_{ij}} \colon \widehat{a} > 0\,\right\} $

\[
  \text{Min}\left\{\,\frac{5}{5}, \frac{12}{4}, \frac{4}{1}\,\right\} = \frac{5}{5} \Longrightarrow x_6 \text{ sai da base }
\]

Os cálculos são feitos com o objetivo de transformar a coluna do $ x_2 $ na 
coluna do $ x_6 $.
Assim, o encontro da coluna da variável que deve entrar na base com a linha da
variável básica que sai da base, $ x_6 $, é o \textcolor{red}{5}.
Este ``5'' é dito \textbf{pivô}.
 
\begin{table}[!h]
\centering
\caption{$\text{Min}\{\,\frac{1}{1/5}, \frac{8}{11/5}, \frac{3}{9/5}\,\} = \frac{3}{9/5} \Rightarrow x_7 \text{ sai da base}$}
\begin{tabular}{|c|cccccccc|c|}
  \hline
  Min                    & \textcolor{red}{$x_1$}   & $x_2$ & $x_3$ & $x_4$ & $x_5$ & $x_6$ & $x_8$ &  w  &    \\
  \hline
  Var. Bas.              &                    -9/5  & 0     & -1/5  & 0     & 1     & 6/5   & 0     & -1  & -3 \\
  \hline 
  $x_2$                  &                    1/5   &   1   &  -1/5 &     0 &   0   &   1/5 &   0   &   0 &  1 \\
  $x_4$                  &                   11/5   &   0   &   4/5 &     1 &   0   &  -4/5 &   0   &   0 &  8 \\
  \textcolor{red}{$x_7$} &   \textcolor{red}{9/5}   &   0   &  -1/5 &     0 &  -1   &  -1/5 &   1   &   0 &  3 \\
  \hline
\end{tabular}
\end{table} 
 
\begin{table}[!h]
\centering
\caption{Problema original é factível}
\begin{tabular}{|c|cccccccc|c|}
  \hline
  Min         & $x_1$ & $x_2$ & $x_3$ & $x_4$ & $x_5$ & $x_6$ & $x_7$ & $w$ &    \\
  \hline
  Var. Bas.   &   0   &   0 &   0 &   0 &   0 &   1 &   1 &  -1 &  0 \\
  \hline
  $x_2$       &   0   &   1 &  -2/9 &   0 &    1/9 &    2/9 &   -1/9 &   0 &  2/3 \\
  $x_4$       &   0   &   0 &   5/9 &   1 &   11/9 &   -5/9 &  -11/9 &   0 & 13/3 \\
  $x_1$       &   1   &   0 &   1/9 &   0 &   -5/9 &   -1/9 &    5/9 &   0 &  5/3 \\
  \hline
\end{tabular}
\end{table}

A solução básica factível atual é ótima e $ w = 0 $ indica que o problema 
original é factível. \hfill (\textbf{Fim da Fase I})

Agora se tem uma solução básica factível do problema original: $ x_2 = \frac{2}{3} $,
$ x_4 = \frac{13}{3} $ e $ x_1 = \frac{5}{3} $.
Com estes vetores, $ z = \frac{26}{3} $.

Agora é a vez de montar o quadro para \textbf{Fase II}.

Retomando a F.O. original: \newpage

\begin{table}[!h]
\centering
\caption{Quadro ainda não preparado}
\begin{tabular}{|c|cccccc|c|}
  \hline
  Min         & $x_1$ & $x_2$ & $x_3$ & $x_4$ & $x_5$ & $z$  &      \\
  \hline
  Var. Bas.   &  4    &   3   &     0 &   0   &     0 &  -1  & 0    \\
  \hline
  $x_2$       &  0    &   1   &  -2/9 &   0   &   1/9 &   0  & 2/3  \\
  $x_4$       &  0    &   0   &   5/9 &   1   &  11/9 &   0  & 13/3 \\
  $x_1$       &  1    &   0   &   1/9 &   0   &  -5/9 &   0  & 5/3  \\
  \hline
\end{tabular}
\end{table}

Falta preparar o quadro para poder fazer a discussão da solução ótima.

\begin{table}[!h]
\centering
\caption{Quadro preparado}
\begin{tabular}{|c|cccccc|c|}
  \hline
  Min                    & $x_1$ & $x_2$ & $x_3$ & $x_4$ & \textcolor{red}{$x_5$} & $z$  &       \\
  \hline
  Var. Bas.              &  0    &   0   &  -2/9 &   0   &  17/9                  &  -1  & -26/3 \\
  \hline
  $x_2$                  &  0    &   1   &  -2/9 &   0   &   1/9                  &   0  &   2/3 \\
  \textcolor{red}{$x_4$} &  0    &   0   &   5/9 &   1   &  \textcolor{red}{11/9} &   0  &  13/3 \\
  $x_1$                  &  1    &   0   &   1/9 &   0   &                   -5/9 &   0  &   5/3 \\
  \hline
\end{tabular}
\end{table}



Quadro preparado!
A solução básica factível atual não é ótima.

Fazendo da equação de bloqueio para se saber se há outra solução básica que
melhore o valor da F.O.

\begin{table}[!h]
\centering
\begin{tabular}{|c|cccccc|c|}
  \hline
  Min         & $x_1$ & $x_2$ & $x_3$ & $x_4$    & $x_5$ & $z$  &         \\
  \hline
  Var. Bas.   &  0    &   0   & -3/11 & -17/11   &     0 &  -1  & -169/11 \\
  \hline
  $x_2$       &  0    &   1   & -3/11 &  -1/11   &     0 &   0  &    3/11 \\
  $x_5$       &  0    &   0   &  5/11 &   9/11   &     1 &   0  &   39/11 \\
  $x_1$       &  1    &   0   &  4/11 &   5/11   &     0 &   0  &   40/11 \\
  \hline
\end{tabular}
\end{table}

A solução básica factível atual é ótima: 
\[ 
  x_2 = \frac{3}{11} ,  \quad x_5 = \frac{39}{11} \quad \text{ e } \quad x_1 = \frac{40}{11} .
\]
O valor ótimo de $ z $ é 

\begin{center}
  \shadowbox
  {
    $ \dfrac{169}{11} $
  }	
\end{center}
